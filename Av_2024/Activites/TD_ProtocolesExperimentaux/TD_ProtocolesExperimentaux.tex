\documentclass[10pt,fleqn]{article} % Default font size and left-justified equations
\usepackage[%
    pdftitle={Hyperstatisme},
    pdfauthor={Xavier Pessoles}]{hyperref}

    
\input{style/new_style}
\input{style/macros_SII}
\usepackage{multicol}
\usepackage{siunitx}
%\usepackage{picins}
\fichetrue
%\fichefalse

\proftrue
\proffalse

\tdtrue
%\tdfalse

\courstrue
\coursfalse


\def\classe{\textsf{PSI$\star$}}
\def\xxnumpartie{Cycle xx}
\def\xxpartie{Préparation aux épreuves orales}

\def\xxnumchapitre{Filière PSI \vspace{.2cm}}
\def\xxchapitre{\hspace{.12cm} }


\def\discipline{Sciences \\Industrielles de \\ l'Ingénieur}
\def\xxtete{Sciences Industrielles de l'Ingénieur}


\def\xxactivite{TD}
\def\xxauteur{\textsl{Xavier Pessoles}}


\def\xxtitreexo{Systèmes de TP}
\def\xxsourceexo{\hspace{.2cm} \footnotesize{Laboratoire de PSI}}


  
\def\xxposongletx{2}
\def\xxposonglettext{1.45}
\def\xxposonglety{20}
%\def\xxonglet{Part. 1 -- Ch. 3}
\def\xxonglet{\textsf{Cycle xx}}

\def\xxactivite{TD}
\def\xxauteur{\textsl{Xavier Pessoles}}

\def\xxcompetences{%
\vspace{-.5cm}
\footnotesize{
\textsl{%
%\textbf{Savoirs et compétences :}\\
%\vspace{-.2cm}
%\begin{itemize}[label=\ding{112},font=\color{ocre}] 
%\item \textit{Mod2.C34} : chaînes de solides;
%\item \textit{Mod2.C34} : degré de mobilité du modèle;
%\item \textit{Mod2.C34} : degré d’hyperstatisme du modèle;
%\item \textit{Mod2.C34.SF1} : déterminer les conditions géométriques associées à l’hyperstatisme;
%\item \textit{Mod2.C34} : résoudre le système associé à la fermeture cinématique et en déduire le degré de mobilité et d’hyperstatisme.
%\end{itemize}
}}}


\def\xxfigures{
%\includegraphics[width=.8\textwidth]{images/fig_01}
}%figues de la page de garde


\def\xxpied{%
Préparation aux oraux\\%dans le but de déterminer les contraintes géométriques dans les mécanismes\\% afin de valider leurs performances.\\
%Chapitre 2 -- \xxactivite%
}

\setcounter{secnumdepth}{5}
%---------------------------------------------------------------------------


\begin{document}
%\chapterimage{png/Fond_Cin}
\input{style/new_pagegarde}
\vspace{4.5cm}
\pagestyle{fancy}
\thispagestyle{plain}


\def\columnseprulecolor{\color{ocre}}
\setlength{\columnseprule}{0.4pt} 

\begin{multicols}{2}

En utilisant le système de votre choix, proposer les protocoles expérimentaux permettant de répondre aux objectifs suivants.
\begin{enumerate}
\item Déterminer l'inertie d'un solide en rotation.
\item Déterminer l'inertie équivalente ramenée à l'arbre moteur d'un moteur à courant continu.
\item Déterminer la résistance d'un moteur à courant continu.
\item Déterminer l'inductance d'un moteur à courant continu.
\item Déterminer la constante électrique d'un moteur à courant continu.
\item Déterminer les frottements secs.
\item Déterminer les frottements visqueux.
\end{enumerate}

En utilisant le système de votre choix, proposer les protocoles expérimentaux permettant de répondre aux objectifs suivants.
\begin{enumerate}
\item Déterminer le modèle de comportement d'un système modélisable par un système d'ordre 1.
\item Déterminer le modèle de comportement d'un système modélisable par un système d'ordre 1 intégré.
\item Déterminer le modèle de comportement d'un système modélisable par un système d'ordre 2.
\item Déterminer une tension de seuil.
\item Déterminer une saturation.
\item Déterminer un rapport de transmission et linéariser autour d'un point de fonctionnement.
\end{enumerate}


\end{multicols}

\end{document}

\subparagraph{}\textit{}
\ifprof
\begin{corrige}~\\
\end{corrige}
\else
\fi

\begin{center}
\includegraphics[width=\linewidth]{images/img_04}
%\textit{}


\end{center}

