\documentclass[10pt,fleqn]{article} % Default font size and left-justified equations
\usepackage[%
    pdftitle={Hyperstatisme},
    pdfauthor={Xavier Pessoles}]{hyperref}

    
\input{style/new_style}
\input{style/macros_SII}
\usepackage{multicol}
\usepackage{siunitx}
%\usepackage{picins}
\fichetrue
%\fichefalse

\proftrue
\proffalse

\tdtrue
%\tdfalse

\courstrue
\coursfalse


\def\classe{\textsf{PSI$\star$}}
\def\xxnumpartie{Cycle xx}
\def\xxpartie{Préparation aux épreuves orales}

\def\xxnumchapitre{Filière PSI \vspace{.2cm}}
\def\xxchapitre{\hspace{.12cm} }


\def\discipline{Sciences \\Industrielles de \\ l'Ingénieur}
\def\xxtete{Sciences Industrielles de l'Ingénieur}


\def\xxactivite{TD}
\def\xxauteur{\textsl{Xavier Pessoles}}


\def\xxtitreexo{Systèmes de TP}
\def\xxsourceexo{\hspace{.2cm} \footnotesize{Laboratoire de PSI}}


  
\def\xxposongletx{2}
\def\xxposonglettext{1.45}
\def\xxposonglety{20}
%\def\xxonglet{Part. 1 -- Ch. 3}
\def\xxonglet{\textsf{Cycle xx}}

\def\xxactivite{TD}
\def\xxauteur{\textsl{Xavier Pessoles}}

\def\xxcompetences{%
\vspace{-.5cm}
\footnotesize{
\textsl{%
%\textbf{Savoirs et compétences :}\\
%\vspace{-.2cm}
%\begin{itemize}[label=\ding{112},font=\color{ocre}] 
%\item \textit{Mod2.C34} : chaînes de solides;
%\item \textit{Mod2.C34} : degré de mobilité du modèle;
%\item \textit{Mod2.C34} : degré d’hyperstatisme du modèle;
%\item \textit{Mod2.C34.SF1} : déterminer les conditions géométriques associées à l’hyperstatisme;
%\item \textit{Mod2.C34} : résoudre le système associé à la fermeture cinématique et en déduire le degré de mobilité et d’hyperstatisme.
%\end{itemize}
}}}


\def\xxfigures{
%\includegraphics[width=.8\textwidth]{images/fig_01}
}%figues de la page de garde


\def\xxpied{%
Préparation aux oraux\\%dans le but de déterminer les contraintes géométriques dans les mécanismes\\% afin de valider leurs performances.\\
%Chapitre 2 -- \xxactivite%
}

\setcounter{secnumdepth}{5}
%---------------------------------------------------------------------------


\begin{document}
%\chapterimage{png/Fond_Cin}
\input{style/new_pagegarde}
\vspace{4.5cm}
\pagestyle{fancy}
\thispagestyle{plain}


\def\columnseprulecolor{\color{ocre}}
\setlength{\columnseprule}{0.4pt} 

\begin{multicols}{2}
\subsection*{BGR}
\setcounter{exo}{0}

\begin{center}
\includegraphics[width=\linewidth]{images/bgr_01}
%\textit{}
\end{center}

\subparagraph{}
\textit{Proposer un schéma cinématique 2D puis 3D.}

\subparagraph{}
\textit{Donner le degré d’hyperstatisme du modèle proposé.}


\subparagraph{}
\textit{Paramétrer le mécanisme.}


\subparagraph{}
\textit{Réaliser la chaîne d'information et la chaîne d'énergie.}






\subsection*{Direction Assistée Électrique}
\setcounter{exo}{0}
\begin{center}
\includegraphics[width=\linewidth]{images/dae_02}
%\textit{}
\end{center}

\subparagraph{}
\textit{Réaliser les schémas cinématiques 2D puis 3D associés à la DAE.}

\subparagraph{}
\textit{Donner le degré d’hyperstatisme du modèle proposé.}

\subparagraph{}
\textit{En réalisant les hypothèses adéquates, proposer une relation entre le couple au volant et le couple de rotation des roues.}


\subparagraph{}
\textit{En réalisant les hypothèses adéquates, proposer une relation entre le couple moteur et le couple de rotation des roues (le moteur est relié à la colonne de direction via un réducteur roue et vis sans fin.}


\subparagraph{}
\textit{Réaliser la chaîne d'information et la chaîne d'énergie.}


\subsection*{Barrière Sympact}
\setcounter{exo}{0}
\begin{center}
\includegraphics[width=\linewidth]{images/sympact_01}
%\textit{}
\end{center}

\subparagraph{}
\textit{Réaliser le schéma cinématique 2D.}

\subparagraph{}
\textit{Réaliser le paramétrage.}


\subparagraph{}
\textit{Déterminer la loi entrée -- sortie.}



\subparagraph{}
\textit{Déterminer la relation entre le couple moteur, les caractéristiques du ressort, le poids de la barrière et les caractéristiques géométriques. }

\subparagraph{}
\textit{Réaliser la chaîne d'information et la chaîne d'énergie.}


\subsection*{MaxPID}


\setcounter{exo}{0}
\begin{center}
\includegraphics[width=\linewidth]{images/maxpid_01}
%\textit{}
\end{center}

\subparagraph{}
\textit{Réaliser le schéma cinématique 2D.}

\subparagraph{}
\textit{Donner le degré d’hyperstatisme du modèle proposé.}


\subparagraph{}
\textit{Réaliser le paramétrage.}


\subparagraph{}
\textit{Déterminer la loi entrée -- sortie.}

\subparagraph{}
\textit{Déterminer la relation entre le couple moteur, les caractéristiques géométriques et massiques. }


\subparagraph{}
\textit{Réaliser la chaîne d'information et la chaîne d'énergie.}

\subsection*{Cheville du robot NAO}


\setcounter{exo}{0}
\begin{center}
\includegraphics[width=\linewidth]{images/nao_02}
%\textit{}
\end{center}

\begin{center}
\includegraphics[width=\linewidth]{images/nao_01}
%\textit{}
\end{center}



\subparagraph{}
\textit{Réaliser le schéma cinématique 3D permettant d'illustrer le roulis et le tangage.}


\subparagraph{}
\textit{Réaliser le schéma cinématique de la chaîne de transmission.}


\subparagraph{}
\textit{Réaliser la chaîne d'information et la chaîne d'énergie.}


\subsection*{Pilote hydraulique de voilier}
\setcounter{exo}{0}
\begin{center}
\includegraphics[width=\linewidth]{images/pilote}
%\textit{}
\end{center}
On considère le bâti, le corps du vérin, la tige du vérin et le bras de mèche. 

\subparagraph{}
\textit{Réaliser le schéma cinématique 2D.}

\subparagraph{}
\textit{Donner le degré d’hyperstatisme du modèle proposé.}


\subparagraph{}
\textit{Réaliser le paramétrage.}


\subparagraph{}
\textit{Déterminer la loi entrée -- sortie.}

\end{multicols}

\end{document}

\subparagraph{}\textit{}
\ifprof
\begin{corrige}~\\
\end{corrige}
\else
\fi

\begin{center}
\includegraphics[width=\linewidth]{images/img_04}
%\textit{}


\end{center}

